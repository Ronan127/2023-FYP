\chapter{Planned Activities}

\section{Review Literature}
Study work by previous students on LIMed robots. Additionally, research existing solutions for USAR platforms and identify methods that are applicable to this project. 

\section{2D Simulation}
Perform a simple 2D simulation of a LIMed robot platform to gain a greater understanding of how it functions.

\section{Create Model}
Create a mathematical model to describe the kinematics of a LIMed robot platform. This will be presented in a way that describes how the degrees of freedom will move as a function of the platform's state and the torque applied to it.

\section{Compile Design Requirements}
Determine the requirements for the system, including functions that the platform must be able to perform and the cost of the platform. List these requirements in a way that they can be used to evaluate different concepts.

\section{Design Platform}
Compare concepts for a LIMed robot platform, then select the preferred concept based on its ability to meet the design requirements. Then create a detailed design of the robot platform.

\section{Simulate Platform}
Perform a mature simulation of the designed platform to ensure it will meet the requirements.

\section{Produce Prototype}
Construct a prototype of the robot platform. This will involve a combination of manufacture by the MMW, and assembly of the components.

\section{Validate Model}
Test that the model accurately describes the kinematics of a LIMed system through experimentation using the prototype.

\section{Finalise Report}
Document the entire project, including the process of creating a model, designing and production of the prototype, and experimentation. Comment on whether the model can be used to accurately describe the LIM system.\\\\

\noindent Please see Appendix A for the budget and time allocation of these activities.