\chapter{Risk Assessment}

The device uses a maximum voltage of 12 V, which is considered safe to touch. Gears are enclosed within the frame of the LIMs, however power should still be disconnected when handling the device as the geared motors can produce significant torque. To prevent motor burnout, they should not be left stalling for more than a few seconds at a time.\\

\noindent The main risk to the feasibility of the project was that there would not be enough time to build a fully functioning robot platform. The projects preceding this one all encountered complications when building robot platforms that they were unable to solve within the time constraints of their projects. This risk was mitigated by completing activities in a timely manner as described in the Gantt. Chart, as well as careful planning and consideration of previous pitfalls.
\\

\noindent There was also the risk that it would be impossible to create a closed form model to describe the dynamics of the system. Fortunately, this was not the case.\\

\noindent There was a risk that the project would exceed the allowed cost. This is because the prototype may have required hardware such as high powered motors, custom parts, encoders, controllers, and possibly batteries, which could have a considerable cost. This was be mitigated simplifying the design to only use the minimum hardware required to build a prototype.\\

