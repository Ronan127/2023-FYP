\chapter{Executive summary}

\noindent
\begin{longtable}{|p{\dimexpr \linewidth-2\tabcolsep-2\arrayrulewidth}|}
\hline%------------------------------------------------------------
\sumheading  Title of Project \\
\hline%------------------------------------------------------------
 Design, model and build a USAR robot platform \\
\hline%------------------------------------------------------------
\sumheading  Objectives \\
\hline%------------------------------------------------------------
 Create a model to describe the kinematics of a Load Intuitive Module (LIM). \\
 Build a prototype Urban Search and Rescue (USAR) device which uses LIMs to climb stairs.\\
 Validate the model using the prototype.\\
\hline%------------------------------------------------------------
\sumheading  What is current practice and what are its limitations? \\
\hline%------------------------------------------------------------
 The current practice for USAR platform ranges widely, but the most successful platforms use tracks with paddles for locomotion.
 These devices are effective but very expensive, so there is a need for low cost expendable USAR robots.\\
\hline%------------------------------------------------------------
\sumheading  What is new in this project? \\
\hline%------------------------------------------------------------
 This project will introduce a model to describe a less expensive stair climbing robot platform using LIMs. \\
 
\hline%------------------------------------------------------------
\sumheading  If the project is successful, how will it make a difference? \\
\hline%------------------------------------------------------------
 The model developed in this project can be used to inform future USAR designs. \\

\hline%------------------------------------------------------------
\sumheading  What are the risks to the project being a success? Why is it expected to be successful? \\
\hline%------------------------------------------------------------
 The main risk to this project is that it does not build a working prototype in time. This risk will be mitigated through careful planning and consideration of previous pitfalls.  \\

\hline%------------------------------------------------------------
\sumheading  What contributions have/will other students made/make? \\
\hline%------------------------------------------------------------
 In 2013, Matthew Wilson developed the LIM system as a masters project at the University of Cape Town (UCT).
 Further development on the system was done in final year projects at UCT by students Jordan Haskel, Murray Buchanan, and Richard Daniel Powrie in 2017, 2018, and 2019 respectively.\\
\hline%------------------------------------------------------------
\sumheading  Which aspects of the project will carry on after completion and why? \\
\hline%------------------------------------------------------------
 USAR devices using LIMs as a platform can be designed, built and tested. \\

\hline%------------------------------------------------------------
\sumheading  What arrangements have been/will be made to expedite continuation? \\
\hline%------------------------------------------------------------
 All calculations, designs, and code will be made available to future students. \\

\hline%------------------------------------------------------------
\end{longtable}

