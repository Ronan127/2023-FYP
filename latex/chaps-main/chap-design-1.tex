\chapter{Design- 1st iteration}

Starting with Powrie's device as it is the most developed of the previous projects.\\
Attempted to use mathematical model to identify potential improvements, however I struggled to get objective improvements, increasing one parameter often may improve the ability to flip but hinder the ability to climb overall, and removing material should only be done if it has minimal effect on structure strength, something that would take much time to determine (possibly with FEM?). Such optimisations are beyond the scope of the project, I'm not trying to create a perfectly optimised device, I'm trying to create a working device so I can describe its function. Powrie’s device is working to some extent, so the first design iteration should deviate minimally from his design.\\
Powrie's design for the LIMs is copied as accurately as possible, could possibly even use his laser cutting templates to manufacture. Fortunately he provides a detailed builder's guide, allowing me to design and build ASAP so I can focus on the math model. The robot body is changed significantly. Powrie uses external gears on an already geared motor, this is unnecessary, as geared motors come in a variety of ratios. I use a JGY-370 motor, as I believe the worm gearbox is well suited to this purpose, it allows me to make a thinner body that doesn't protrude as far forward beyond the axle. The tail is designed based on Powrie’s concepts. The motors selected produce more torque than the ones Powrie selected, even with his additional gearing, so they should be up to the task. Later iterations could combine the worm gearbox with even more powerful brushless motors.