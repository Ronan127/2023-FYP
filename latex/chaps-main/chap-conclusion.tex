\chapter{Conclusions}

This report details the design, modelling, and model validation of a novel gearing system for robot locomotion. The advantage of this LIM gearing system is that it allows a device to roll forward or climb steps using a single actuator, reducing costs, while also being able to fit into low voids. The model produced by this report could be used to inform the design of future USAR robots. This project builds and improves upon previous work, and produces the first LIM robot capable of climbing steps consistently and sequentially. \\

To explore the motion of the LIMs, a preliminary two-dimensional simulation was performed, and the stair climbing motion of the device was categorised into six distinct stages. This project proposes a mathematical model to describe the motion a LIMed device, using the MATLAB symbolic toolbox. The maths model consists of a large set of simultaneous equations that are solved simultaneously, and this report presents an algorithm that uses the MATLAB functions to efficiently solve this large set of equations. Additionally, this project explores the use of Drake, a multi-body simulator, to model the device. \\

Both the maths model and simulation are validated against real world data. The validation showed that the torque required to climb stairs sequentially was 16.4\% higher than the maths model predicted, and 5.9\% higher than the simulation predicted. This report explores the reasons for this error, which are attributed to imperfect assumptions and calibration methods.\\

In conclusion, this project was successful in completing its objectives. Both the model and simulation have been validated and their error has been quantified, they can be used to inform future work on LIMed robots.