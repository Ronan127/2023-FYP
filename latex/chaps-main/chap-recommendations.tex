\chapter{Recommendations for future work}

This project was successful in building a LIMed device that can climb stairs. However, the device does have some limitations. It cannot turn easily, and it relies on an external power source. The logical next steps in the development of LIM robots would be to find a way to improve the turning ability of the device, and to design a device that can be operated remotely. Such a device could then be tested for applicability in USAR environments.


\section{Recommendation for design}

When designing a device that uses LIMs, this project recommends first using the maths model to inform the selection design parameters. This can be done by changing a parameter and noting how it affects the performance of the design, similar to what was done in Figure \ref{fig:GR-friction}. Once the design parameters have been selected, the device should be designed in CAD and converted to SDFormat. From there the device can be simulated using Drake, and if performance is satisfactory, it can be built. Instructions on how to perform each of these steps can be found in the project repository \citep{repo}.\\
