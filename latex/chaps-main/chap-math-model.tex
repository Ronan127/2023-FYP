\chapter{Maths Model}

Overview and objectives\\

This project develops a system of equations to analytically model the climbing motion of the device, which will be referred to as the, "Maths Model" henceforth. The intention behind this model is to allow designers to input parameters such as gear ratio, mass, wheel size, tail length, and motor torque, then determine whether the specified device will be able to climb steps. The equations can also reversed to solve for specific parameters, such as the motor torque required to lift a device with certain properties. The maths model is developed using the MATLAB symbolic toolbox.\\

definition of motions\\

As a LIMed device climbs, the tail and wheels come into contact with different surfaces on the stairs and ground, which changes how the device moves. The overall climbing motion is broken down into sequential stages which are modelled individually. These stages are defined as follows:\\
\subsection*{Stage 0: Rolling}

When the LIMed device does is on a flat plane with no obstacle, it simply rolls forward. If the motor torque is high enough, the LIMs will flip even without an obstacle. However, DC motors lose torque as they gain speed, so the rolling motion prevents the motors from producing enough torque to flip the LIMs.\\

\subsection*{Stage 1: Lifting}
When the front wheel of the LIM comes into contact with the first step, it is blocked by a step and fixed in place. The tail pushes against the ground and the LIM starts rotating up the step. This motion ends when the LIM is vertical.
\\
\subsection*{Stage 2: Flipping}

Once the LIM is vertical, the bottom wheel starts to roll backwards as the top wheel falls forward onto the step. The distance that the bottom wheel rolls depends on the speed of the LIM and the height of the step. If the frame of the LIM hits the edge of the step, the device may slip backwards until the top wheel makes contact with the step. This motion ends when the top wheel is on the step.\\

\subsection*{Stage 3: Climbing}

The front wheel rests on the step while the back wheel is on the ground. The tail pushes against the ground and the back wheel lifts while the front wheel simultaneously rolls forward on the step. This motion ends when the front wheel reaches the next step.\\

Stages 1 to 3 will repeat until the tail leaves the ground. After this point the tail will push against the edge of the previous steps. 

\subsection*{Stage 4: Lifting from step}

Similar to Stage 1, but the tail pushes against the edge of a the previous step, which now applies a force pulling the device backwards. Assuming friction on the wheel is sufficient, the LIM starts rotating up the step. This motion ends when the LIM is vertical.

\subsection*{Stage 5: Flipping from step}

Similar to Stage 2, but the tail pushes against the edge of a the previous step, which now applies a force pulling the device backwards. This motion ends when the top wheel is on the step.\\

\subsection*{Stage 6: Climbing from step}

Similar to Stage 3, but the tail pushes against the edge of a the previous step, which now applies a force pulling the device backwards. The back wheel lifts while the front wheel rolls forward on the step. This motion ends when the front wheel reaches the next step; however, due to the tail force pulling the device backwards, the back wheel will continue to lift past the horizontal while front wheel rolls forward.\\


Core equations and assumptions\\
Boundary conditions\\
Solving technique\\
Required torque\\
Required coefficient of friction\\

The device can climb up steps. In doing so, the wheels and tail make contact with different parts of the steps. In order to model the device, the overall movement is broken down into individual sequential motions.\\
The first motion is simple, the device rolls forward on a flat surface.\\
The second motion is referred to as climbing. The front wheel is blocked by a step and fixed in place. The tail pushes against the ground and the LIM starts rotating up the step. This motion ends when the LIM is vertical.\\
In the third motion, the top wheel falls forward onto the step and the bottom wheel rolls backwards until the top wheel lands. The distance that it rolls depends on the speed of the LIM and the height of the step. If the frame of the LIM hits the edge of the step, the device may slip backwards until the top wheel makes contact with the step. This motion ends when the top wheel is on the step.\\
In the fourth motion, the device pulls itself up the steps. The front wheel rests on the step while the back wheel is on the ground. The tail pushes against the ground and the bottom wheel lifts while the front wheel simultaneously rolls forward on the step. This motion ends when the top wheel reaches the next step.\\
The fifth motion is similar to the second motion, with the exception that the tail is angled further down to reach the ground.\\
The sixth motion is similar to the third motion, with the exception that the tail is angled further down to reach the ground.\\
The seventh motion deviates significantly from the previous motions. The front wheel starts on the next step while the bottom wheel starts on the previous step. The tail now contacts the edge of the previous step rather than the ground, which causes a significant force to pull the LIMs backwards. Because of this, unlike in the fourth motion, the front wheel will not initially roll forward on the next step. The back wheel will lift from below until it is at a certain angle above the front wheel, at which point the front wheel will start to roll forward, bringing it to the base of the next step.\\
The eighth motion is similar to the second motion, except the tail pushes against the edge of the previous step and the back wheel is already partially lifted. The back wheel then lifts up further until the LIM is vertical.\\
The ninth motion is similar to the sixth motion, except the tail pushes against the surface of the previous step instead of the ground.