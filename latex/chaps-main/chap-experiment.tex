\chapter{Experiment}
In order to validate that the model is accurate for each motion, the torque that causes the device to perform each motion determined experimentally. The equivalent values produced by can be compared with the measured torques to validate the model quantitatively. However, the torque produced by a DC motor cannot be measured or set directly. Instead, The voltage across the motor is varied. The torque of a DC motor is proportional to the current running through it, and the stalling torque is proportional to the voltage. To characterise the motors, an experiment was set up to determine the stalling torque produced at different input voltages. \\
\\
A lever was placed on the output shaft of the motor gearbox, and a mass was attached to the end of the lever. The voltage of the motor was varied and the distance that it lifts the weight was recorded at each voltage.\\
Experiment:\\\\

Independent variable:\\
$\bullet$ Motor voltage (V)\\
Dependent variable:\\
$\bullet$ Distance the weight is lifted (mm)\\
Controlled variables:\\
$\bullet$ Motor used\\
$\bullet$ Lever used\\
$\bullet$ Power supply used\\
$\bullet$ Multimeter used\\
$\bullet$ Ruler used\\


Now that the relation between motor voltage and stalling torque has been calculated, the torque required to perform each motion 